\nnchapter{ВВЕДЕНИЕ}
\aftertitle

В настоящее время в мире генерируется большое количество различной информации. Большинство людей используют множество источников информации и следят за огромным количеством новостного контента. В дополнение к традиционным новостным агентствам всё большую роль в генерации новостного трафика приобретают социальные медиа.

Многие пользователи сталкиваются с патологической зависимостью от чтения новостных и социальных лент. Данная зависимость в настоящее время является предметом изучения социальной и психологической науки \cite{news_addiction}. Отдельной формой зависимости от чтения новостей получило явление зависимости от новостей негативного характера, таких как эпидемии, катастрофы и т.п., названное в зарубежных социальных медиа и литературе doomscrolling \cite{doomscrolling} и получившее известность в период пандемии COVID-19.

Зависимость от чтения новостных лент несет ряд негативных факторов для человека. В первую очередь, это непосредственная потеря времени, потраченного на просмотр новостных источников. Кроме того, это приводит к повышенной утомляемости, проблемам с концентрацией внимания, низкой трудоспособностью. Человек сталкивается с невозможностью продуктивно выполнять трудовую деятельность и полноценно отдыхать, что еще сильнее усугубляет негативный эффект.

В последнее время все большую популярность получает стремление к продуктивности, включающее в себя информационную гигиену, осознанный отказ от потребления избыточного количества зачастую ненужной информации. Люди стараются потреблять меньше контента новостного и развлекательного характера, ограничивать время чтения новостей, получать только важную информацию.

Тем не менее, многие люди не хотят полностью лишаться новостной и другой интересующей их информации. Одним из выходов является использование различных подборок главных новостных дайджестов. Недостатком этого способа является его ограниченность. Подборки новостей могут быть ограничены по своей тематике и источникам, например, не включать интересующие пользователя темы, и включать информацию, которую пользователь сочтет не интересной и не нужной.

Схожей и также актуальной на сегодняшний день проблемой, связанной с избыточным количеством социальной и новостной информации, является возросшая популярность групповых чатов как способов не только простого общения, но и обсуждения тех или иных вопросов. На смену интернет-форумам, пик популярности которых пришелся на 1990-2010 года, приходят групповые чаты. Существует огромное количество тематических чатов, посвященных самым разнообразным темам, таким как путешествия, программирование, поиск работы, домашние животные, учеба, местные городские события и т.д.

Специфика общения в групповых чатах сильно отличается от интернет-форумов. Традиционно, формой общения на форумах были относительно большие и редкие комментарии, более активная модерация, множество отдельных тем и асинхронная форма общения. В отличие от них, групповые чаты зачастую предполагают живое общение в реальном времени, большое количество маленьких сообщений, один общий чат, менее строгую модерацию и, следовательно, большое количество флуда, т.е. сообщений, не относящихся к тематике чата.

Основной проблемой использования групповых чатов как средства получения информации является огромный поток сообщений. В зависимости от уровня активности, групповой чат с несколькими тысячами пользователей может генерировать тысячи сообщений в день. В таком количестве сообщений очень трудно искать и получать интересующую пользователя информацию.

В настоящий момент не представлено решений, позволяющих в полной мере решить описанные проблемы.

Поэтому, актуальной является задача разработки системы, осуществляющей персонализированную фильтрацию и пересказ интересующей пользователя информации из новостных лент и групповых чатов.

Для выполнения цели были поставлены следующие задачи:
\begin{enumerate}
    \item произвести аналитический обзор предметной области с целью выяснения актуального состояния, существующих решений, их преимуществ и недостатков, существующих подходов к решению поставленной цели;
    \item спроектировать бота персональной фильтрации для Telegram, позволяющего пользователю составлять персональный дайджест новостей на основе информации выбранных каналов и чатов;
    \item реализовать бота персональной фильтрации;
    \item произвести исследования с использованием разработанного бота.
\end{enumerate}

В первой главе приведен обзор аналогов и анализ существующих подходов к построению рекомендательных систем с использованием больших языковых моделей.
