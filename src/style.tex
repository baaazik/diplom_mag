% TODO: 1. Длинные заголовки при переносе строки продолжаются под цифрами. Т.е. не
%          начало текста не выровнено
% ----------------------------- БАЗОВЫЕ НАСТРОЙКИ  -----------------------------

%\usepackage[T2A]{fontenc}
%\usepackage[utf8]{inputenc}                        % Кодировка utf8x для linux
\usepackage[english,russian]{babel}                % Переносы и прочее для русского и английского
\babelfont{rm}{LiberationSerif}

%\usepackage{cmap}                                  % Нормальные русские символы в получаемом pdf
\usepackage[linktocpage=true]{hyperref}            % Гиперссылки
\usepackage{indentfirst}                           % Красная строка для первых абзацев
\usepackage{mathtools}
\usepackage{tabularx}                              % Много табличной боли и магии
\usepackage{multirow}
\usepackage{graphicx}                              % Для вставки картинок
\usepackage{titlesec}                              % Для настройки названий глав, разделов итп
\usepackage{titletoc}                              % Для настройки оглавления
\usepackage{enumitem}                              % Для настройки списков
\usepackage{flafter}                               % Плавающие элементы встречаются после ссылки на них
\usepackage{caption}                               % Настройка подписей плавающих элементов
\usepackage{setspace}                              % Для настройки интервалов
\usepackage{csquotes}
\usepackage{chngcntr}                              % Чтоб настроить сквозную нумерацию
\usepackage{lastpage}                              % Получить количество страниц
\usepackage[figure,table]{totalcount}              % Получить количество рисунков и таблиц
\usepackage{svg}                                   % Использование SVG в рисунках
\usepackage{pdfpages}
\usepackage{listings}
\usepackage{fancyhdr}
\usepackage{tempora}

% Библиография
% CREDITS: https://github.com/AndreyAkinshin/Russian-Phd-LaTeX-Dissertation-Template

%%% Реализация библиографии пакетами biblatex и biblatex-gost с использованием движка biber %%%
% Это особое колдунство позволяет сделать так, что в начале идет только один автор
% Фамилия, И. О. Название / И. О. Фамилия1, И. О. Фамилия 2 // ...

\usepackage{csquotes} % biblatex рекомендует его подключать. Пакет для оформления сложных блоков цитирования.
%%% Загрузка пакета с основными настройками %%%
\makeatletter
% Чистовик
%----------------
\usepackage[%
backend=biber,% движок
bibencoding=utf8,% кодировка bib файла
sorting=none,% настройка сортировки списка литературы
style=gost-numeric,% стиль цитирования и библиографии (по ГОСТ)
language=autobib,% получение языка из babel/polyglossia, default: autobib % если ставить autocite или auto, то цитаты в тексте с указанием страницы, получат указание страницы на языке оригинала
autolang=other,% многоязычная библиография
clearlang=true,% внутренний сброс поля language, если он совпадает с языком из babel/polyglossia
defernumbers=true,% нумерация проставляется после двух компиляций, зато позволяет выцеплять библиографию по ключевым словам и нумеровать не из большего списка
sortcites=true,% сортировать номера затекстовых ссылок при цитировании (если в квадратных скобках несколько ссылок, то отображаться будут отсортированно, а не абы как)
doi=false,% Показывать или нет ссылки на DOI
isbn=false,% Показывать или нет ISBN, ISSN, ISRN
]{biblatex}[2016/09/17]
\ltx@iffilelater{biblatex-gost.def}{2017/05/03}%
{\toggletrue{bbx:gostbibliography}%
\renewcommand*{\revsdnamepunct}{\addcomma}}{}
%----------------

\makeatother

\providebool{blxmc} % biblatex version needs and has MakeCapital workaround
\boolfalse{blxmc} % setting our new boolean flag to default false

%http://tex.stackexchange.com/a/141831/79756
%There is a way to automatically map the language field to the langid field. The following lines in the preamble should be enough to do that.
%This command will copy the language field into the langid field and will then delete the contents of the language field. The language field will only be deleted if it was successfully copied into the langid field.
\DeclareSourcemap{ %модификация bib файла перед тем, как им займётся biblatex
    \maps{
        \map{% перекидываем значения полей language в поля langid, которыми пользуется biblatex
            \step[fieldsource=language, fieldset=langid, origfieldval, final]
            \step[fieldset=language, null]
        }
        \map{% перекидываем значения полей numpages в поля pagetotal, которыми пользуется biblatex
            \step[fieldsource=numpages, fieldset=pagetotal, origfieldval, final]
            \step[fieldset=numpages, null]
        }
        \map{% перекидываем значения полей pagestotal в поля pagetotal, которыми пользуется biblatex
            \step[fieldsource=pagestotal, fieldset=pagetotal, origfieldval, final]
            \step[fieldset=pagestotal, null]
        }
        \map[overwrite]{% перекидываем значения полей shortjournal, если они есть, в поля journal, которыми пользуется biblatex
            \step[fieldsource=shortjournal, final]
            \step[fieldset=journal, origfieldval]
            \step[fieldset=shortjournal, null]
        }
        \map[overwrite]{% перекидываем значения полей shortbooktitle, если они есть, в поля booktitle, которыми пользуется biblatex
            \step[fieldsource=shortbooktitle, final]
            \step[fieldset=booktitle, origfieldval]
            \step[fieldset=shortbooktitle, null]
        }
        \map{% если в поле medium написано "Электронный ресурс", то устанавливаем поле media, которым пользуется biblatex, в значение eresource.
            \step[fieldsource=medium,
            match=\regexp{Электронный\s+ресурс},
            final]
            \step[fieldset=media, fieldvalue=eresource]
            \step[fieldset=medium, null]
        }
        \map[overwrite]{% стираем значения всех полей issn
            \step[fieldset=issn, null]
        }
        \map[overwrite]{% стираем значения всех полей abstract, поскольку ими не пользуемся, а там бывают "неприятные" латеху символы
            \step[fieldsource=abstract]
            \step[fieldset=abstract,null]
        }
        \map[overwrite]{ % переделка формата записи даты
            \step[fieldsource=urldate,
            match=\regexp{([0-9]{2})\.([0-9]{2})\.([0-9]{4})},
            replace={$3-$2-$1$4}, % $4 вставлен исключительно ради нормальной работы программ подсветки синтаксиса, которые некорректно обрабатывают $ в таких конструкциях
            final]
        }
        \map[overwrite]{ % стираем ключевые слова
            \step[fieldsource=keywords]
            \step[fieldset=keywords,null]
        }
        % реализация foreach различается для biblatex v3.12 и v3.13.
        % Для версии v3.13 эта конструкция заменяет последующие 7 структур map
        % \map[overwrite,foreach={authorvak,authorscopus,authorwos,authorconf,authorother,authorparent,authorprogram}]{ % записываем информацию о типе публикации в ключевые слова
        %     \step[fieldsource=$MAPLOOP,final=true]
        %     \step[fieldset=keywords,fieldvalue={,biblio$MAPLOOP},append=true]
        % }
        \map[overwrite]{ % записываем информацию о типе публикации в ключевые слова
            \step[fieldsource=authorvak,final=true]
            \step[fieldset=keywords,fieldvalue={,biblioauthorvak},append=true]
        }
        \map[overwrite]{ % записываем информацию о типе публикации в ключевые слова
            \step[fieldsource=authorscopus,final=true]
            \step[fieldset=keywords,fieldvalue={,biblioauthorscopus},append=true]
        }
        \map[overwrite]{ % записываем информацию о типе публикации в ключевые слова
            \step[fieldsource=authorwos,final=true]
            \step[fieldset=keywords,fieldvalue={,biblioauthorwos},append=true]
        }
        \map[overwrite]{ % записываем информацию о типе публикации в ключевые слова
            \step[fieldsource=authorconf,final=true]
            \step[fieldset=keywords,fieldvalue={,biblioauthorconf},append=true]
        }
        \map[overwrite]{ % записываем информацию о типе публикации в ключевые слова
            \step[fieldsource=authorother,final=true]
            \step[fieldset=keywords,fieldvalue={,biblioauthorother},append=true]
        }
        \map[overwrite]{ % записываем информацию о типе публикации в ключевые слова
            \step[fieldsource=authorpatent,final=true]
            \step[fieldset=keywords,fieldvalue={,biblioauthorpatent},append=true]
        }
        \map[overwrite]{ % записываем информацию о типе публикации в ключевые слова
            \step[fieldsource=authorprogram,final=true]
            \step[fieldset=keywords,fieldvalue={,biblioauthorprogram},append=true]
        }
        \map[overwrite]{ % добавляем ключевые слова, чтобы различать источники
            \perdatasource{biblio/external.bib}
            \step[fieldset=keywords, fieldvalue={,biblioexternal},append=true]
        }
        \map[overwrite]{ % добавляем ключевые слова, чтобы различать источники
            \perdatasource{biblio/author.bib}
            \step[fieldset=keywords, fieldvalue={,biblioauthor},append=true]
        }
        \map[overwrite]{ % добавляем ключевые слова, чтобы различать источники
            \perdatasource{biblio/registered.bib}
            \step[fieldset=keywords, fieldvalue={,biblioregistered},append=true]
        }
        \map[overwrite]{ % добавляем ключевые слова, чтобы различать источники
            \step[fieldset=keywords, fieldvalue={,bibliofull},append=true]
        }
%        \map[overwrite]{% стираем значения всех полей series
%            \step[fieldset=series, null]
%        }
        \map[overwrite]{% перекидываем значения полей howpublished в поля organization для типа online
            \step[typesource=online, typetarget=online, final]
            \step[fieldsource=howpublished, fieldset=organization, origfieldval]
            \step[fieldset=howpublished, null]
        }
    }
}

% Кодунство со StackOverflow для того, чтобы текст не выравнивался под цифрами
% https://tex.stackexchange.com/questions/598595/no-indentation-in-non-second-lines-of-bibliography-list
\setlength{\biblabelsep}{\labelsep}
\defbibenvironment{bibliography}
  {\list
     {\printtext[labelnumberwidth]{%
        \printfield{labelprefix}%
        \printfield{labelnumber}}}
     {\setlength{\labelwidth}{-\labelnumberwidth}%
      \setlength{\labelsep}{\biblabelsep}%
      \setlength{\leftmargin}{0pt}%
      \setlength{\itemindent}{\bibhang}%
      \setlength{\itemsep}{\bibitemsep}%
      \setlength{\parsep}{\bibparsep}}%
      \renewcommand*{\makelabel}[1]{\hss##1}}
  {\endlist}
  {\item}


\linespread{1.5}                                   % Полуторный интервал

\usepackage[
	left=3cm,
	right=1cm,
    top=2cm,
	bottom=2cm,
	bindingoffset=0cm]{geometry}


\sloppy                                           % Избегать залезания строк на поля (надо?)
\setlength\parindent{1.25cm}                      % Отступ красной строки

\newcommand{\nf}{\normalfont}

% Для tabularx:
\newcolumntype{Y}{>{\centering\arraybackslash}X} % Растянутый столбец (как X) с выравниванием по центру
\newcolumntype{P}{>{\raggedleft\arraybackslash}X}% Растянутый столбец (как X) с выравниванием по справа

% НАСТРОЙКИ НУМЕРАЦИИ СТРАНИЦ
% Нумерация страницу снизу справа

% Задаем наш стиль
\fancypagestyle{gost}{
	% Очистка текущих колонтитулов
	\fancyhf{}
	% Установить колонтитул страницы
	\fancyfoot[R]{\thepage}
	% Удаление линии разделителя колонтитулов
	\renewcommand{\headrulewidth}{0pt}
	% Отступ между верхним колонтитулом и текстом
	\addtolength{\headsep}{-0.5cm}
	% Отступ между нижним колонтитулом и текстом
	\addtolength{\footskip}{-0.1cm}
}

% \chapter переключает первую страницу в стиль "plain", поэтому наш стиль не
% применяется
\fancypagestyle{plain}[gost]{}

\pagestyle{gost}


% ----------------------------- НАСТРОЙКИ ЗАГЛАВИЙ -----------------------------
%% Отступ 1.5 слева (как у красной строки)t
% Нет точки между номером и названием
% Интервал между подзаголовками 1.5
% Интервал между заголовком и текстом 2*1.5
% Поддержка приложений

% Глава
\titleformat{\chapter}
	[block]                % Shape. block убирает перенос заглвания на новую строку
    {\normalfont}          % Format. Собственно, стиль
    {\thechapter}          % Label. Номер главы.
    {8pt}                  % Sep. Пробел между номером и главой (TODO: уточнить)
    {}                     % before-code. Код перед названием
\titlespacing*{\chapter}
	{1.25cm}               % Левый отступ (как у красной строки)
	{18pt}                 % Верхний отступ, 1.5 интервал
	{18pt}                 % Нижний отступ, 1.5 интервал

% Раздел
\titleformat{\section}
	{\normalfont}
	{\thesection}
	{8pt}{}
\titlespacing*{\section}
	{1.25cm}{18pt}{18pt}

% Подраздел
\titleformat{\subsection}
	{\normalfont}
	{\thesubsection}
	{8pt}{}
\titlespacing*{\subsection}
	{\parindent}{18pt}{18pt}

% Глава без номера (введение, заключение и т.п.)
\newcommand{\nnchapter}[1]
{
	\chapter*{#1}
	\addcontentsline{toc}{chapter}{#1}
}

% Приложения
% Использовать \chapter{} для создания приложений
% Очень грязный хак, но работает
\newcommand{\StartAppendix}
{
	\setcounter{chapter}{0}
}

\renewcommand{\appendix}[1]
{
	\newpage
	\stepcounter{chapter}
	\newcommand{\theappendix}{ПРИЛОЖЕНИЕ \MakeUppercase{\asbuk{chapter}}}
	\addcontentsline{toc}{chapter}{\texorpdfstring{\theappendix} ~--- #1}
	\begin{center}
		\theappendix\\
		{#1}
	\end{center}
}

% Расстояние между заглавиями и текстом должно быть 2 полуторных интервала,
% а расстояние между заглавиями - один полуторный интервал.
% Не придумал ничего лучше, кроме как вставлять вручную
\newcommand{\aftertitle}{\vskip 18pt}

% ----------------------------- НАСТРОЙКИ СОДЕРЖАНИЯ ---------------------------
% Нет выделения жирным
% Все с одним уровнем отступа
% Поддержка приложений

% Главы
\titlecontents{chapter}
	[0em] {}
	{\thecontentslabel~}{}
	{\titlerule*[1pc]{.}\contentspage}

% Разделы
\titlecontents{section}
	[0em] {}
	{\thecontentslabel~}{}
	{\titlerule*[1pc]{.}\contentspage}

% Подразделы
\titlecontents{subsection}
	[0em] {}
	{\thecontentslabel~}{}
	{\titlerule*[1pc]{.}\contentspage}

% Заголовок
\addto\captionsrussian{
	\renewcommand{\contentsname} {СОДЕРЖАНИЕ}
}

%-------------------------------- НАСТРОЙКИ СПИСКОВ ----------------------------
% Маркерный список
\setlist[itemize]{
	label=-,                  % Дефис в каяестве маркера
	leftmargin=1.25cm,         % Текст в списке выравнен по красной строке
	itemindent=15pt,          % Маркер выравнен по красной строке, т.е. первая строка чуть сдвинута на размер маркера
	nosep                     % Убираем интервал между пунктами списков
}

% Числовой
\setlist[enumerate]{
    leftmargin=1.25cm,
    nosep
}
\setlist[enumerate,1]{
	label=\arabic*),
	itemindent=18pt,
}
\setlist[enumerate,2]{
	label=\arabic{enumi}.\arabic*),
	leftmargin=0cm,
	itemindent=30pt,
}
% NOTE: сейчас поддерживается только два уровня вложенности, если нужно больше -
% нужно добавить

%--------------------------- НАСТРОЙКИ РИСУНКОВ И ТАБЛИЦ -----------------------
% Рисунки подписываются "Рисунок N - ..." по центру
% Таблицы подписываются "Таблица N - ..." с левого края

\captionsetup[figure]{
	name=Рисунок,
	labelsep=endash,
	justification=centering,
	belowskip=-17pt,aboveskip=0pt,
	font={stretch=1.3}
}
\captionsetup[table]{name=Таблица, labelsep=endash, justification=raggedright, singlelinecheck=false}

% Сквозная нумерация таблиц, рисунков и формул
\counterwithout{figure}{chapter}
\counterwithout{table}{chapter}
\counterwithout{equation}{chapter}
%\pdfimageresolution=150

%-------------------------------- БИБЛИОГРАФИЯ ---------------------------------

%\addbibresource{bibliography.bib}
